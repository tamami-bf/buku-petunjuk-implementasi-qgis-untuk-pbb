\chapter{MEMBUAT DATA SPASIAL (DIGITASI)}

Sebelumnya kita telah belajar bagaimana membuat peta sederhana dengan menampilkan Data Spasial yang telah disediakan. Tetapi, kita juga harus mempelajari bagaimana membuat Data Spasial yang baru, terutama jika kita tidak mempunyai Data Spasial tersebut. Misalkan, kita mempunyai sumber data \textit{raster} seperti citra satelit, foto udara, peta rupa bumi Indonesia, atau peta lainnya yang memiliki informasi koordinat, kita dapat membuat data spasialnya dengan melakukan digitasi terhadap data \textit{raster} tersebut.

