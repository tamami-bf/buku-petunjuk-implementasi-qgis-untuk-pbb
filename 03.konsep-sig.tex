

\chapter{KONSEP DASAR SISTEM INFORMASI GEOGRAFIS (SIG)}

\begin{enumerate}[A.]
\item{Definisi}

Sistem Informasi Geografis (Geographic Information System / GIS) yang selanjutnya disebut SIG merupakan sistem informasi berbasis komputer yang digunakan untuk mengolah dan menyimpan data atau informasi geografis.

Secara umum pengertian SIG adalah sebagai berikut :
\begin{italicquotes}
Suatu komponen yang terdiri dari \textbf{perangkat keras, perangkat lunak, sumber daya manusia, dan data} yang bekerja bersama secara efektif untuk memasukkan, menyimpan, memperbaiki, memperbaharui, mengelola, memanipulasi, mengintegrasikan, menganalisa, dan menampilkan data dalam suatu informasi berbasis geografis.
\end{italicquotes}

Dalam pembahasan selanjutnya, SIG akan selalu diasosiasikan dengan sistem yang berbasis komputer, walaupun pada dasarnya SIG dapat dikerjakan secara manual, SIG yang berbasis komputer akan sangat membantu ketika data geografis merupakan data yang besar (dalam jumlah dan ukuran) dan terdiri dari banyak tema yang saling berkaitan.

SIG mempunyai kemampuan untuk menghubungkan berbagai data pada suatu titik tertentu di bumi, menggabungkannya, menganalisa, dan akhirnya memetakan hasilnya. Data yang akan diolah pada SIG merupakan data spasial, yaitu sebuah data yang berorientasi geografis dan merupakan lokasi yang memiliki sistem koordinat tertentu, sebagai dasar referensinya. Sehingga aplikasi SIG dapat menjawab beberapa pertanyaan seperti: lokasi, kondisi, trend, pola, dan pemodelan. Kemampuan inilah yang membedakan SIG dari sistem informasi lainnya.

\item{Komponen SIG}

Secara umum SIG bekerja berdasarkan integrasi 4 (empat) komponen, yaitu : \textit{hardware}, \textit{software}, manusia, dan data.

\begin{enumerate}[1.]
\item \textit{Hardware} / Perangkat Keras

SIG membutuhkan \textit{hardware} atau perangkat komputer yang memiliki spesifikasi lebih tinggi dibandingkan dengan sistem informasi lainnya untuk menjalankan \textit{software-software} SIG, seperti kapasitas \textit{memory} (RAM), \textit{harddisk}, prosesor, serta \textit{VGA Card}. Hal tersebut disebabkan karena data-data yang digunakan dalam SIG baik data vektor maupun data raster penyimpanannya membutuhkan ruang yang besar dan dalam proses analisanya membutuhkan \textit{memory} yang besar dan prosesor yang cepat.

\item \textit{Software} / Perangkat Lunak

\textit{Software} SIG merupakan sekumpulan program aplikasi yang dapat memudahkan kita dalam melakukan berbagai macam pengolahan data, penyimpanan, \textit{editing}, hingga \textit{layout}, ataupun analisis keruangan.

\item Sumber Daya Manusia

Teknologi SIG tidaklah menjadi bermanfaat tanpa manusia yang mengelola sistem dan membangun perencanaan yang dapat diaplikasikan sesuai kondisi dunia nyata. Sama seperti pada Sistem Informasi lain, pemakai SIG pun memiliki tingkatan tertentu, dan tingkat spesialis teknis yang mendesain dan memelihara sistem sampai pada penguna yang menggunakan SIG untuk menolong pekerjaan mereka sehari-hari.

\item Data

Data dan informasi spasial merupakan bahan dasar dalam SIG. Data ataupun realitas di dunia / alam akan diolah menjadi suatu informasi yang terangkum dalam suatu sistem berbasis keruangan dengan tujuan-tujuan tertentu.
\end{enumerate}

Telah dijelaskan di awal bahwa SIG adalah suatu kesatuan sistem yang terdiri dari berbagai komponen, tidak hanya perangkat keras komputer beserta dengan perangkat lunaknya saja akan tetapi harus tersedia data geografis yang benar dan sumber daya manusia untuk melaksanakan perannya dalam memformulasikan dan menganalisa persoalan yang menentukan keberhasilan SIG.

Tingkat keberhasilan dari suatu kegiatan SIG dengan tujuan apapun itu sangat bergantung dari interaksi ke empat faktor ini. Jika salah satunya pincang maka hasilnya pun tidak akan ada gunanya.

\item Konsep Layer Data

Konsep layer data adalah representasi data spasial menjadi sekumpulan peta tematik yang berdiri sendiri-sendiri sesuai dengan tema masing-masing, tetapi terkait dalam suatu kesamaan lokasi. Keuntungan dari konsep data layer adalah memungkinkan kita melakukan penelusuran data dan analisa data dengan mudah serta efisiensi dalam pengolahan data. Sedangkan atribut merupakan nilai data ataupun informasi yang terangkum pada suatu lokasi. Misalnya, suatu lokasi bencana disimbolkan dengan titik, maka informasi atau data yang ada pada lokasi tersebut akan diberi nama atribut.

\item Model Aplikasi SIG

\item Bidang Kebencanaan

\item Bidang Kesehatan

\item Bidang Perencanaan Pembangunan

\end{enumerate}
